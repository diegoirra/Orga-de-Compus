\documentclass[10pt,a4paper]{article}
\usepackage[utf8]{inputenc}
\usepackage{amsmath}
\usepackage{amsfonts}
\usepackage{amssymb}
\usepackage{listings}
\usepackage{mips}
\usepackage[spanish]{babel}
\usepackage{color}

\definecolor{dkgreen}{rgb}{0,0.6,0}
\definecolor{gray}{rgb}{0.5,0.5,0.5}
\definecolor{mauve}{rgb}{0.58,0,0.82}

\begin{document}

\date{14 de Noviembre del 2017}

\title{Univesidad de Buenos Aires - FIUBA \\ 66:20 Organización de Computadoras \\ Trabajo práctico 1: Programación MIPS}

\author{Bacigaluppo, Ivan (98064) \\ Irrazabal, Diego (98125)}

\maketitle

\thispagestyle{empty}

\newpage

\setcounter{page}{1}

\tableofcontents

\newpage

\section{Documentación}

\subsection{Diseño}

Debido a la mayor complejidad de este segundo trabajo practico, nos vimos obligados cambiar la forma de trabajo y nuestras ideas de modelo incremental. Realizando una mayor cantidad de modulos y probando cada uno de forma independente con un main de prueba.

La primera base en C era igual al tp anterior, mas ya sin la función de palindromo implementada. El código C sufrió dos cambios no importantes. Se implementó un algoritmo que cicla sobre los argumentos para detectar los correctos, pudiendo estos estar en desorden, y habiendo aumentado en número. Una vez establecidas las variables pasadas por parámetro o default, se llama a una función handle para manejarlas. Esta función se encarga de abrir y cerrar los archivos, llamando en el medio al segundo cambio significativo de este archivo, la función palindromo. Esta posee un firma distinta, ya que ahora además de detectar si una palabra es un palindromo, debe leer y escribir en los archivos, creando y utilizando los buffers correspondientes, todo implementado en el lenguaje de assembler mips.

Para las funciones en mips, intentamos probar por su cuenta la mayor cantidad posible. Comenzando por las más básicas syscall de lectura y escritura, implementadas dentro de las funciones getch y putch. Getch() se encarga de leer el archivo medianta una syscall, llenando un buffer de entrada con los datos. Mientas que putch() se encarga de escribir los datos del buffer de salida en el archivo correspondiente.
Esta parte fue en su nivel más fácil y rápida puesto que la cátedra nos facilitó los ejemplos apropiados para trabajar con estas syscall.

La cátedra también nos proveyó con las funciones mymalloc() y myfree() implementadas en mips para crear y liberar los buffers de memoria. A su vez, la funcion que contendria el algoritmo para detectar si una palabra es un palíndromo también fue rápida, puesto que era básicamente traducir el algoritmo diseñado en C para el primero trabajo practico.

Luego implementamos las funciones que necesitaríamos para el diseo final del programa. Getwordlen() adquiere un puntero del bufer de entrada y cicla por sus caracteres hasta encontrar aquel que no pertence a los aceptados por la consigna, obteniendo así la longitud de esta palabra. Con este dato la palabra es copiada al buffer interno del programa, y luego procesado por el algoritmo palindromo. De ser palíndromo esta palabra se copia al buffer de salida. Estas copias se realizan con otra función llamada strcopy(), que copia una determinada cantidad de bytes de una cadena a otra. Luego se vuelve a ciclar sobre el buffer de entrada para leer la siguiente palabra.

Por último, la integración. Habiendo probando las funciones de forma independente, uno pensaría que esta parte sería bien facil. Pues uno estaría equivocado. El mayor problema que nos trajo la integración de las funciones la cantidad de variables que nos fuimos obligados a guardar. Por esto, debimos crear dos funciones locales que nos aseguraran no perder los valores de estas variables al llamar a otra funcion, o subrutina. Luego, sí pudimos regresar a nuestro diseño incremental donde integrabamos de a poco cada función, lo que trajo problemas a la hora de procesar los ciclos, ya que debido a la sintaxis de mips es muy díficil ver a simple vista los ciclos anidados, asi como sus condiciones de corte, o quizas ni siquiera. Las llamadas a etiquetas desde branches tambien dificultan la identificacion de ciclos, puesto que tienen la misma sintaxis que el reconocimiento de las condiciones de corte.

Dos rutinas más fueron creadas en este proceso. Print es llamada cuando el bufer de salida fue lleno y resume el proceso de llamar a putch. Por otro lado, se impementó clear_buf para vaciar los buffers cuando esten llenos o dspues de procesar la palabra para el itnerno.

\newpage

\subsection{Implementación}

\subsubsection{void print\_version()}

Imprime la version del programa por consola.

\subsubsection{void print\_usage()}

Imprime la leyenda de ayuda del programa por consola.

\subsubsection{int palindrome(int ifd, size\_t ibytes, int ofd, size\_t obytes)}

Recibe un input file descriptor, un tamaño de bytes para los buffer de lectura, un output file descriptor y un tamañao de bytes para los buffers de salida. Internamente llama a la función implementada en MIPS is\_pal.

\subsubsection{void handle(char* input\_file\_name, char* output\_file\_name, int input\_buffer, int output\_buffer,)}

Recibe punteros a los nombres completos de los archivos de entrada y salida, procede a ejecutar la logica completa del programa, mediante el llamado a la función de MIPS (si algun nombre es equivalente a "-" se reemplaza por la entrada/salida estandar). A su vez, recibe los tamaños de los buffer de entrada y salida.

\subsubsection{int main(int argc, char** argv)}

Punto de entrada al programa, se procesan los parametros recibidos de linea de comando y se ejecuta la logica del programa de ser correctos, de lo contrario se muestra la leyenda de ayuda.

\subsubsection{int getch(int, int, char*)}

Recibe el file descriptor del archivo de entrada abierto, la cantidad bytes a leer (el tamaño del buffer) y el buffer donde almacenar los datos. Devuelve un entero correspondiente a la canidad de datos leidos, es decir, debiendo ser igual al tamaño del buffer. Si ocurre algun error devuelve un codigo acorde. Entiéndase 0 si se llegó al final del archivo, -1 si hubo un error al leer el archivo, y -2 si los bytes leidos no son los esperados.

\subsubsection{int putch(int, int, char*)}

Recibe el file descriptor del archivo de salida abierto, la cantidad bytes a imprimir (el tamaño del buffer) y el buffer de donde adquirir los datos. Devueve la cantidad de bytes que se imprimieron en el archivo, o un código de error acorde: -1 si el error fue con el archivo en la syscall, y -2 si fuera el tamaño de bytes impresos.


\subsubsection{int getwordlen(char*)}

Recibe el puntero al indice actual del buffer de entrada, y cicla controlando que los caracteres sean los aprobados como parte de una palabra. Corta al encontrar alguno que no, y devuelve la longitud de la palabra qe se encontró. Este dato deberá ser sumado al puntero para en el siguiente ciclo leer la siguiente palabra, y no la misma.


\subsubsection{int strcopy(int, char*, char*)}

Copia la cantidad de bytes pasada por parámetro del primer puntero de caracteres al segundo. Devolviendo el número de bytes que copió para controlar.


\subsubsection{int is_pl(char*)}

El algoritmo para detectar si una palabra es un palindromo devuelve 1 si lo es y 0 en caso contrario.


\subsubsection{void clear_buf(char*)}

Se encarga de vaciar el buffer pasado por parámetro. Ciclando por él, y guardando el caracter null 0 en cada posición que tenga un dato.


\subsection{Compilación}

\lstset{
	breaklines=true,
  	basicstyle=\footnotesize,
}

Se ha incluido un archivo Makefile para simplificar la obtención del ejecutable, el mismo puede obtenerse simplemente mediante la ejecución del comando \textit{make}, que generará un archivo binario \textit{tp1}:

\begin{lstlisting}[language=bash]
$ make
gcc -g -Wall -o tp1 tp1.c mymalloc.S is_pal.S getch.S getwordlen.S putch.S palindrome.S strcopy.S clear_buf.S
$ ls
tp1.c Makefile tp1 
\end{lstlisting}

\subsection{Ejecución}

Una vez obtenido el ejecutable, el mismo se puede ejecutar con el parámetro \textit{-h} para obtener la leyenda de ayuda:

\begin{lstlisting}[language=bash]
$ ./tp1 -h
Usage:
	tp1 -h
	tp1 -V
	tp1 [options]
Options:
	-V, --version	Print version and quit.
	-h, --help	Print this information.
	-i, --input	Location of the input file.
	-o, --output	Location of the output file.
  -I, --ibuf-bytes  Byte-count of the input buffer.
  -O, --obuf-bytes  Byte-count of the output buffer.
Examples:
	tp1 -i ~/input -o ~/output

\end{lstlisting}

O utilizando cualquiera de los parámetros requeridos por el enunciado:

\begin{lstlisting}[language=bash]
$ echo "somos todos bob" | ./tp1 -o pal.txt
$ cat pal.txt
somos
bob
\end{lstlisting}

\newpage

\section{Casos de prueba}

\subsection{Makefile}

Se puede utilizar el Makefile para correr casos de prueba:

\begin{lstlisting}[language=bash]
$ make test
gcc -g -Wall -o tp1 tp1.c 
./tp1 -i tests/test1.in > tests/test1.res
diff tests/test1.out tests/test1.res
./tp1 -i tests/test2.in > tests/test2.res
diff tests/test2.out tests/test2.res
./tp1 -i tests/test3.in > tests/test3.res
diff tests/test3.out tests/test3.res
./tp1 -i tests/test4.in > tests/test4.res
diff tests/test4.out tests/test4.res
\end{lstlisting}

\subsection{Validación de parámetros}

\subsubsection{Input}

\begin{lstlisting}[language=bash]
$ ./tp1 -i /tmp/noexiste.txt
No se pudo abrir el archivo de entrada: /tmp/noexiste.txt
\end{lstlisting}

\subsubsection{Output}

\begin{lstlisting}[language=bash]
$ echo "bob" |./tp1 -o /tmp/
No se pudo abrir el archivo de salida: /tmp/
\end{lstlisting}

\subsection{Utilizado entrada y salida standard}

\begin{lstlisting}[language=bash]
$ echo "somos bob hope"|./tp1 
somos
bob
\end{lstlisting}

\subsection{Utilizado archivos}

\begin{lstlisting}[language=bash]
$ echo "somos bob hope" >> test.txt
$ ./tp1 -i test.txt -o pal.txt
$ cat pal.txt 
somos
bob
\end{lstlisting}

\newpage

\section{Código fuente}

\subsection{C}

\lstset{
	breaklines=true,
  	basicstyle=\footnotesize,
	numbers=left,
	captionpos=b,
	showspaces=false,
	title=\lstname
}

\lstinputlisting[language=C]{tp1.c}

\newpage

\subsection{Assembly (MIPS)}

\lstset{
	breaklines=true,
  	basicstyle=\footnotesize,
}

\begin{lstlisting}[language=bash]
$ make asm
gcc -g -Wall -O0 -S tp1.c mymalloc.S is_pal.S getch.S getwordlen.S putch.S palindrome.S strcopy.S clear_buf.S
\end{lstlisting}

\lstset{
  language=[mips]Assembler,       % the language of the code
  basicstyle=\footnotesize,       % the size of the fonts that are used for the code
  numbers=left,                   % where to put the line-numbers
  numberstyle=\tiny\color{gray},  % the style that is used for the line-numbers
  stepnumber=1,                   % the step between two line-numbers. If it's 1, each line 
  numbersep=5pt,                  % how far the line-numbers are from the code
  backgroundcolor=\color{white},  % choose the background color. You must add \usepackage{color}
  showspaces=false,               % show spaces adding particular underscores
  showstringspaces=false,         % underline spaces within strings
  showtabs=false,                 % show tabs within strings adding particular underscores
  frame=single,                   % adds a frame around the code
  rulecolor=\color{black},        % if not set, the frame-color may be changed on line-breaks within not-black text (e.g. commens (green here))
  tabsize=4,                      % sets default tabsize to 2 spaces
  captionpos=b,                   % sets the caption-position to bottom
  breaklines=true,                % sets automatic line breaking
  breakatwhitespace=false,        % sets if automatic breaks should only happen at whitespace
  title=\lstname,                 % show the filename of files included with \lstinputlisting;
  keywordstyle=\color{blue},          % keyword style
  commentstyle=\color{dkgreen},       % comment style
  stringstyle=\color{mauve},         % string literal style
  escapeinside={\%*}{*)},            % if you want to add a comment within your code
  morekeywords={*,...}               % if you want to add more keywords to the set
}

\lstinputlisting{palindrome.S}

\section{Conclusiones}

Dado que el trabajo no está terminado no hay muchas conclusiones para obtener. Lo que pudimos observar es que entender el lenguaje MIPS no es simple y lleva tiempo y trabajo. Esto se puede ver en que algo que fue muy simple en código C, como obtener palíndromos de un archivo, fue totalmente distinto al pasar a código fuente, en donde nos encontramos con muchas dificultades. A parte de esto, pudimos seguir familiarizandonos con herramientas como LaTex y la máquina virtual de gxemul, la cual nos permite levantar la imagen de NetBSD, sin la cual no podríamos trabajar con código MIPS.


\end{document}
